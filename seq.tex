\let\negthickspace\undefined
\documentclass[journal,12pt,twocolumn]{IEEEtran}
\usepackage{cite}
\usepackage{amsmath,amssymb,amsfonts,amsthm}
\usepackage{algorithmic}
\usepackage{graphicx}
\usepackage{textcomp}
\usepackage{xcolor}
\usepackage{txfonts}
\usepackage{listings}
\usepackage{enumitem}
\usepackage{mathtools}
\usepackage{gensymb}
\usepackage{comment}
\usepackage[breaklinks=true]{hyperref}
\usepackage{tkz-euclide} 
\usepackage{listings}
\usepackage{gvv}                                        
\def\inputGnumericTable{}                                 
\usepackage[latin1]{inputenc}                                
\usepackage{color}                                            
\usepackage{array}                                            
\usepackage{longtable}                                       
\usepackage{calc}                                             
\usepackage{multirow}                                         
\usepackage{hhline}                                           
\usepackage{ifthen}                                           
\usepackage{lscape}
\setlength{\arrayrulewidth}{0.5mm}
\setlength{\tabcolsep}{18pt}
\renewcommand{\arraystretch}{1.5}
\newtheorem{theorem}{Theorem}[section]
\newtheorem{problem}{Problem}
\newtheorem{proposition}{Proposition}[section]
\newtheorem{lemma}{Lemma}[section]
\newtheorem{corollary}[theorem]{Corollary}
\newtheorem{example}{Example}[section]
\newtheorem{definition}[problem]{Definition}
\newcommand{\BEQA}{\begin{eqnarray}}
\newcommand{\EEQA}{\end{eqnarray}}
\newcommand{\define}{\stackrel{\triangle}{=}}
\theoremstyle{remark}
\newtheorem{rem}{Remark}
\begin{document}
\title{Sequence(19) 10.5.3}
\author{EE23BTECH11051-Rajnil Malviya}
\date{January 2024}
\maketitle
\subsection*{\textit{Question :-}}
200 logs are stacked in the following manner: 20 logs in the bottom row, 19 in the next row,
18 in the row next to it and so on . In how many rows are the 200 logs placed
and how many logs are in the top row?
\begin{table}[h!]
   
        \begin{tabular}{ | m{1.0cm} | m{4cm} | } 
  \hline
 Symbol & Description \\ 
 \hline
 $n$ & term number \\
 \hline
$a_1$& first term(n=1) of A.P \\
\hline
$a_n$ & $n_{th}$ term of A.P\\ 
\hline
 d & common difference of A.P \\
\hline
$S_n $& Sum upto $n_{th}$ of A.P \\
\hline

\end{tabular}\\
\caption{}
\label{Table:1}
       
    \end{table}
\subsection*{\textit{Solution :-}}

For an Arithmetic Progression :-
\begin{align}a_n-a_{n-1} \; = \; d \end{align}
\begin{align}a_{n-1}-a_{n-2} \; = \; d \end{align}
\begin{align}a_{n-2}-a_{n-3} \; = \; d \end{align}
$$.$$
$$.$$
$$.$$
\begin{align}a_2-a_{1} \; = \; d \end{align}
adding all these equations :-
\begin{align}a_{n}-a_{1} \; = \; \brak{n-1} d \end{align}
\begin{align}a_{n} \; = \; a_1+\brak{n-1} d \end{align}
Sum upto n terms of A.P :-
\begin{align}S_n=a_1+a_2+....+a_n\end{align}
\begin{align}S_n=a_n+a_{n-1}+....+a_1\end{align}
adding equations 7 and 8 
\begin{align}2 S_n=(a_1+a_n)+(a_2+a_{n-2})+....+(a_n+a_1)\end{align}
Substituting by using equation 6 \\
$2 S_n=[a_1+(a_1 +(n-1) d)]+ ...$
\begin{align}....+[(a_1 + (n-1) d) + a_1]\end{align}
\begin{align}S_n=\frac{n}{2}\brak{2a_1 + (n-1) d}\end{align}
\begin{table}[h!]
   
        \begin{tabular}{ | m{1.0cm} | m{4cm} | } 
  \hline
 Symbol & Value \\ 
 \hline
$a_1$& 20 (logs in bottom row) \\
\hline
$a_2$ & 19\\ 
\hline
$a_3$ & 18\\ 
\hline
$S_n$& 200 (total number of logs) \\
\hline
\end{tabular}\\
\caption{}
\label{Table:1}
       
    \end{table}
Using equation 4 
\begin{align}d=a_2-a_1\end{align}
Substituting from Table:II
\begin{align}d=-1\end{align}
\begin{table}[h!]
   
        \begin{tabular}{ | m{1.0cm} | m{4cm} | } 
  \hline
 Symbol & Description \\ 
 
 \hline

$a_n$ & number of logs in top most row \\ 
\hline
\end{tabular}\\
\caption{}
\label{Table:1}
       
    \end{table}
For Practical reasons 

    
\begin{align}a_n>0\end{align}
Using equation 13 and substituting in equation 6
\begin{align}20+(n-1)(-1)>0\end{align}
\begin{align}n<21\end{align}
Using equation 11 and substituting from Table:2
\begin{align}n^2-41 n +400 = 0\end{align}
\begin{align}n=16\; ,\;25\end{align}
From equation 16 
\begin{align}n=16\end{align}
Substituting in equation 6
\begin{align}a_n=20+(16-1)(-1)\end{align}
\begin{align}a_n=5\end{align}
Ans . There are 16 rows and 5 logs in top row .
\end{document}
